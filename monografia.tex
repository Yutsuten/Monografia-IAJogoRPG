% Tamanho do texto, tipo de papel e documento
\documentclass[12pt,a4paper]{article}

% PACOTES
% Padrão
\usepackage[utf8]{inputenc}
\usepackage{amsmath}
\usepackage{amsfonts}
\usepackage{amssymb}

% Pacotes para alterar fonte
\usepackage[T1]{fontenc}
\usepackage{uarial}
% Personaliza lista de itens
\usepackage{enumitem}
% Muda fonte dos "captions"
\usepackage[font=small]{caption}
% Permite adicionar imagens
\usepackage{graphicx}
% Pacote para referências
%\usepackage{csquotes}
% Coloca alguns termos do artigo em português (figura, tabela)
\usepackage[brazilian]{babel}
% Permite alterar distância entre linhas
\usepackage{setspace}
% Colocando indentação em todos os parágrafos (por default o primeiro parágrafo da seção não tem)
\usepackage{indentfirst}
% Margens nas páginas
\usepackage[top=3cm, left=3cm, bottom=2cm, right=2cm]{geometry}
% Links do sumário para as respectivas páginas
\usepackage[colorlinks=true,linkcolor=black]{hyperref}
% Visualização de códigos no meio do texto
\usepackage{listings}
% Cores no código
\usepackage{color}
% Para inserir URLs
\usepackage{hyperref}
\makeatletter
\g@addto@macro{\UrlBreaks}{\UrlOrds}
\makeatother
% Para impedir as imagens de irem para lugares que não devem
\usepackage{placeins}
% Usado para centralizar verticalmente em tabelas
\usepackage{array}


% CONFIGURAÇÃO DE VARIÁVEIS
% Mostra numero da pagina no canto superior direito
\pagestyle{myheadings}
% Indentação dos parágrafos
\parindent 30pt
% Distância entre linhas
\onehalfspacing
% Acessando comandos internos
\makeatletter
	% Colocando espaço de "um espaço" entre o número e o título da seção (default é espaço de 1 quad (1em))
	\renewcommand{\@seccntformat}[1]{\csname the#1\endcsname\ }
	% Deixa os itens do sumário com pontos
	\renewcommand*\l@section{\@dottedtocline{1}{1.5em}{3.0em}}
	% Remove indentação no sumário
	\renewcommand*\l@subsection{\@dottedtocline{2}{1.5em}{3.0em}}
	\renewcommand*\l@subsubsection{\@dottedtocline{3}{1.5em}{3.0em}}
\makeatother
% Editando títulos dos índices
\addto\captionsbrazilian{
	\renewcommand{\listfigurename}{\centering LISTA DE FIGURAS}
	\renewcommand{\listtablename}{\centering LISTA DE TABELAS}
	\renewcommand{\contentsname}{\centering SUMÁRIO}
}
% Escolhendo a fonte 
\renewcommand{\familydefault}{\sfdefault}
% Estilo de listas
\renewcommand{\theenumi}{\alph{enumi}}
% Cor da URL
\hypersetup{urlcolor=black}
% Fonte da URL
\urlstyle{sf}


% INFORMAÇÕES DO TRABALHO
\author{MATEUS GONÇALEZ ETTO}
\title{UTILIZAÇÃO DE INTELIGÊNCIA ARTIFICIAL EM JOGO RPG}
\date{\today}


% COMANDOS PERSONALIZADOS
% Fonte da imagem
\newcommand{\source}[1]{\small Fonte: {#1}}
% Citações longas
\def\longcitation{\list{}{\vspace{-6mm} \leftmargin4.0cm \singlespace \small}\item[]}
\let\endlongcitation=\endlist
% Capa: objetivo do trabalho
\def\articleobjective{\list{}{\vspace{1.5cm} \leftmargin7.0cm \singlespace \small}\item[]}
\let\endarticleobjective=\endlist
% Código C#
\definecolor{codegreen}{rgb}{0,0.6,0}
\definecolor{codelightgreen}{rgb}{0.5,1,0.5}
\definecolor{codepurple}{rgb}{0.62,0,0.84}
\definecolor{codeblue}{rgb}{0,0,1}
\definecolor{backcolour}{rgb}{0.95,0.95,0.92}
\lstdefinestyle{mystyle}{
    backgroundcolor=\color{backcolour},   
    commentstyle=\color{codegreen},
    keywordstyle=\color{codeblue},
    numberstyle=\tiny\color{codelightgreen},
    stringstyle=\color{codepurple},
    basicstyle=\footnotesize,
    breakatwhitespace=false,
    breaklines=true,
    captionpos=b,
    keepspaces=true,
    numbers=left,
    numbersep=5pt,
    showspaces=false,
    showstringspaces=false,
    showtabs=false,
    tabsize=2
}
\lstset{literate=
  {á}{{\'a}}1 {é}{{\'e}}1 {í}{{\'i}}1 {ó}{{\'o}}1 {ú}{{\'u}}1
  {Á}{{\'A}}1 {É}{{\'E}}1 {Í}{{\'I}}1 {Ó}{{\'O}}1 {Ú}{{\'U}}1
  {à}{{\`a}}1 {è}{{\`e}}1 {ì}{{\`i}}1 {ò}{{\`o}}1 {ù}{{\`u}}1
  {À}{{\`A}}1 {È}{{\'E}}1 {Ì}{{\`I}}1 {Ò}{{\`O}}1 {Ù}{{\`U}}1
  {ä}{{\"a}}1 {ë}{{\"e}}1 {ï}{{\"i}}1 {ö}{{\"o}}1 {ü}{{\"u}}1
  {Ä}{{\"A}}1 {Ë}{{\"E}}1 {Ï}{{\"I}}1 {Ö}{{\"O}}1 {Ü}{{\"U}}1
  {â}{{\^a}}1 {ê}{{\^e}}1 {î}{{\^i}}1 {ô}{{\^o}}1 {û}{{\^u}}1
  {Â}{{\^A}}1 {Ê}{{\^E}}1 {Î}{{\^I}}1 {Ô}{{\^O}}1 {Û}{{\^U}}1
  {œ}{{\oe}}1 {Œ}{{\OE}}1 {æ}{{\ae}}1 {Æ}{{\AE}}1 {ß}{{\ss}}1
  {ű}{{\H{u}}}1 {Ű}{{\H{U}}}1 {ő}{{\H{o}}}1 {Ő}{{\H{O}}}1
  {ç}{{\c c}}1 {Ç}{{\c C}}1 {ø}{{\o}}1 {å}{{\r a}}1 {Å}{{\r A}}1
  {€}{{\EUR}}1 {£}{{\pounds}}1
}
\lstset{style=mystyle}


% INICIO DO ARTIGO
\begin{document}

% CAPA - INÍCIO
\thispagestyle{empty} % Oculta o número da página
\begin{center}
\makeatletter
	\textbf{UNIVERSIDADE ESTADUAL PAULISTA}\\
	\textbf{FACULDADE DE CIÊNCIAS}\\
	\textbf{DEPARTAMENTO DE COMPUTAÇÃO}\\
	\textbf{BACHARELADO EM CIÊNCIA DA COMPUTAÇÃO}\\

	\vspace{4.5cm} % Adicionando espaço extra
	\textbf{\@author}

	\vspace{4.5cm} % Adicionando espaço extra
	\textbf{\large \@title}
	
	\vspace*{\fill} % Adicionando espaço para escrever em baixo
	BAURU – SP\\
	\the\year
\makeatother
\end{center}
% CAPA - FIM

% FOLHA DE ROSTO - INÍCIO
\newpage % Coloca o conteúdo numa nova página
\thispagestyle{empty} % Oculta o número da página
\begin{center}
\makeatletter
	\textbf{\@author}

	\vspace{4.5cm} % Adicionando espaço extra
	\textbf{\large \@title}
\makeatother
\end{center}

\begin{articleobjective}
	Trabalho de Conclusão de Curso de graduação apresentado à disciplina Projeto e Implementação de Sistemas do curso de Bacharelado em Ciência da Computação da Faculdade de Ciências da Universidade Estadual Paulista "Júlio de Mesquita Filho"{} como requisito para obtenção do título de Bacharel em Ciência da Computação.
	
	\vspace{1.0cm} % Adicionando espaço extra
	Orientadora: Profa. Dra. Simone das Graças Domingues Prado
\end{articleobjective}

\begin{center}
	\vspace*{\fill} % Adicionando espaço para escrever em baixo
	BAURU – SP\\
	\the\year
\end{center}
% FOLHA DE ROSTO - FIM

% FOLHA DE APROVAÇÃO - INÍCIO
\newpage % Coloca o conteúdo numa nova página
\thispagestyle{empty} % Oculta o número da página
\begin{center}
\makeatletter
	\textbf{\@author}

	\vspace{3.0cm} % Adicionando espaço extra
	\textbf{\large \@title}
\makeatother
\end{center}

\begin{articleobjective}
	Trabalho de Conclusão de Curso de graduação apresentado à disciplina Projeto e Implementação de Sistemas do curso de Bacharelado em Ciência da Computação da Faculdade de Ciências da Universidade Estadual Paulista "Júlio de Mesquita Filho"{} como requisito para obtenção do título de Bacharel em Ciência da Computação.
\end{articleobjective}

\begin{center}
	\vspace{1.0cm}
	BANCA EXAMINADORA\\
	
	\vspace{1.0cm}
	\underline{\hspace{8cm}}\\
	Profa. Dra. Simone das Graças Domingues Prado\\
	\vspace{1.0cm}
	\underline{\hspace{8cm}}\\
	(Nome do segundo membro da Banca Examinadora)\\
	\vspace{1.0cm}
	\underline{\hspace{8cm}}\\
	(Nome do terceiro membro da Banca Examinadora)

	\vspace*{\fill} % Adicionando espaço para escrever em baixo
	Bauru, \underline{\hspace{1cm}} de \underline{\hspace{3cm}} de \underline{\hspace{1.5cm}}
\end{center}
% FOLHA DE APROVAÇÃO - FIM

% RESUMO - INÍCIO
\newpage % Coloca o conteúdo numa nova página
\thispagestyle{empty} % Oculta o número da página
\section*{\hfil RESUMO} % \hfil centraliza o título
	\singlespace
	\noindent
	Este projeto trata-se da criação de um protótipo de jogo no estilo RPG em turnos,
	em conjunto com a criação de uma Inteligência Artificial capaz de controlar os personagens,
	assim como de aprender a controlá-los melhor com treinamento.
	Para a criação da Inteligência Artificial,
	foram usados conceitos de Redes Neurais Artificiais e Algorítimo Genético,
	e para a criação do jogo e seus scripts
	foi usado o motor de jogo Unity.
	
	\vspace{0.8cm}
	\noindent
	\textbf{PALAVRAS-CHAVE:} Inteligência Artificial, Unity, jogo RPG
% RESUMO - FIM

\onehalfspacing

% ABSTRACT - INÍCIO
\newpage % Coloca o conteúdo numa nova página
\thispagestyle{empty} % Oculta o número da página
\section*{\hfil ABSTRACT} % \hfil centraliza o título
	\singlespace
	\noindent
	This project comes to creating a turn-based RPG game prototype,
	in conjunction with an Artificial Intelligence that is able to control the characters
	as well as to learn how to control them better with training.
	For the creation of the Artificial Intelligence,
	concepts of Artificial Neural Networks and Genetic Algorithm were used,
	and for the creation of the game and its scripts,
	the Unity game engine was used.
	
	\vspace{0.8cm}
	\noindent
	\textbf{KEY-WORDS:} Artificial Intelligence, Unity, RPG game
% ABSTRACT - FIM

\onehalfspacing

% LISTA DE FIGURAS - INÍCIO
\newpage % Coloca o conteúdo numa nova página
\thispagestyle{empty} % Oculta o número da página
\listoffigures % Cria a lista de figuras
% LISTA DE FIGURAS - FIM

% LISTA DE TABELAS - INÍCIO
\newpage % Coloca o conteúdo numa nova página
\thispagestyle{empty} % Oculta o número da página
\listoftables % Cria a lista de tabelas
% LISTA DE TABELAS - FIM

% SUMÁRIO - INÍCIO
\newpage % Coloca o conteúdo numa nova página
\thispagestyle{empty} % Oculta o número da página
\tableofcontents % Cria o Sumário
% SUMÁRIO - FIM

% INTRODUÇÃO - INÍCIO
\newpage % Coloca o conteúdo numa nova página
\section{INTRODUÇÃO}
	Há ainda muitas pessoas que acreditam que jogos eletrônicos são para crianças ou pessoas desocupadas.
	Talvez isso fosse verdade no milênio passado, mas a realidade vem se mostrando ser bem diferente.
	
	De acordo com uma pesquisa da SuperData,
	o mercado de jogos cresceu 8\% de 2014 a 2015,
	com 61 bilhões de dólares circulando nesta indústria (CNBC, 2016).
	Em 2014, o valor da indústria de jogos já havia ultrapassado o da indústria de música em 20 bilhões,
	e está chegando ao da indústria do filmes (NYTIMES, 2014).
	
	Então é fato, a indústria de jogos está movendo bilhões de dólares pelo mundo,
	já passou do de música e não para de crescer.
	Como diz o gerente de produto da Eletronic Sports League, James Lampkin:
	%no artigo da New York Times:
	"Isto está se expandindo fora de controle"{}
	(NYTIMES, 2014, tradução nossa).
	Tais palavras explicam muito bem o estado atual do mercado de jogos.
	
	E não é só nas vendas de jogos e consoles,
	existem muitos torneios de jogos ocorrendo pelo mundo,
	surgindo uma nova categoria de profissionais que,
	em poucos anos atrás era inimaginável, senão motivo de piada,
	que é a categoria de jogador profissional de jogo eletrônico.
	A área de trabalho já existe e é chamada de Esporte Eletrônico (NYTIMES, 2014).
	
	E mesmo nestes torneios, não é por pouco dinheiro que os jogadores se enfrentam.
	No Campeonato Mundial de 2015 (o quinto da série) de League of Legends,
	foi oferecido 1 milhão de dólares para a equipe vencedora mundial do jogo,
	como pode ser visto nas regras do campeonato\footnote{Regras: \url{https://riot-web-static.s3.amazonaws.com/lolesports/Rule\%20Sets/2015\%20Revised\%20World\%20Championship\%20Rule\%20Set\%20Version\%201\_01.pdf}}.
	
	Os prêmios não param por aí.
	O jogo Dota 2 distribuiu 11 milhões de dólares para os 10 melhores do mundo,
	sendo 5 milhões para os campeões,
	se tornando assim o maior prêmio já oferecido em um torneio de jogo eletrônico (NYTIMES, 2014).
	
	E tem muita gente para assistir a estes campeonatos.
	Nos dados mostrados pela Riot\footnote{Dados disponíveis em: \url{http://www.lolesports.com/en_US/articles/worlds-2015-viewership}}
	sobre o Campeonato Mundial de 2015,
	houveram 334 milhões de telespectadores "únicos"{} durante as 4 semanas do torneio,
	somando 360 milhões de horas de visualizações das partidas ao vivo.
	
	Mas a área de jogos não está chamando apenas a atenção do mercado,
	mas também a de pesquisadores.
	Um exemplo é o desenvolvimento e aplicação de técnicas de Inteligência Artificial (IA) em jogos,
	que de acordo com especialistas,
	existem áreas dentro de IA em jogos que ainda estão inexplorados (YANNAKAKIS; TOGELIUS, 2014).
	
	Em 2007, foi montada pela AiGameDev uma lista dos 10 jogos com Inteligência Artificial mais influentes.
	Um exemplo é um jogo chamado Creatures,
	que implementou aprendizado de máquina em uma simulação interativa
	ao usar Redes Neurais nas criaturas do jogo.
	Outro exemplo é o Halo,
	o jogo que implementou pela primeira vez a "árvore de condutas",
	tecnologia que ficou muito popular na indústria de jogos (AIGAMEDEV, 2007).
	
	Um exemplo de IA em jogo que é descrito em detalhes é o F.E.A.R.,
	um jogo FPS em primeira pessoa
	que criou um sistema dinâmico, coordenado, interessante e desafiador.
	Isto foi feito utilizando um sistema chamado Goal Oriented Action Planning (Planejamento de Ações Orientado a Objetivo),
	que foi construído junto de duas técnicas: Algoritmo A* e Máquina de Estados Finitos.
	Os NPCs possuem uma lista de objetivos,
	então durante o jogo eles buscam o plano que irá completar o objetivo com maior prioridade.
	O planejamento feito é similar ao STRIPS,
	tendo-se a situação atual e quais são as ações necessárias para cumprir o objetivo.
	Além disto, foi implementado uma extensão da conduta individual dos NPCs,
	com uma conduta em equipe.
	No entanto, como a conduta dos NPCs foi criada para minimizar a ameaça a si mesmo,
	os extintos básicos do NPC podem sobrescrever a conduta de equipe
	caso a segunda opção seja muito arriscado para si mesmo (ORKIN, 2006).
	
	Percebe-se, desta forma, que a área de jogos está muito ativa e em pleno crescimento,
	tanto no mercado quando em pesquisas.
	Existem áreas inexploradas de IA em jogos,
	com novas fronteiras a serem exploradas.
	Com isto em mente, este trabalho foi desenvolvido,
	%aumentando o número de trabalhos que estão explorando esta área,
	e espera-se contribuir com a comunidade acadêmica e/ou mercado de alguma forma.

	\FloatBarrier
	\subsection{Objetivos do Trabalho}
	
		\FloatBarrier
		\subsubsection{Objetivo Geral}
			Produzir um jogo RPG em turnos que implementa conceitos avançados de Inteligência Artificial,
			sendo esta inteligência capaz de tomar decisões de forma autônoma sobre o que deve fazer,
			assim como ser capaz de aprender a tomar melhores decisões por meio de treinamento.
		
		\FloatBarrier
		\subsubsection{Objetivos Específicos}
			\begin{enumerate}[noitemsep]
				\item Criar um jogo RPG razoavelmente complexo.
				\item Criar uma Inteligência Artificial capaz de jogar o jogo tão bem quanto um ser humano.
				\item Criar uma Inteligência Artificial capaz de aprender conforme joga.
			\end{enumerate}			
	
	\FloatBarrier
	\subsection{Organização da Monografia}
		Este trabalho está dividido em 5 seções, sendo esta seção (Introdução) a primeira. As outras seções são:
		\begin{itemize}[noitemsep]
     		\item Seção 2, \textbf{Ferramentas Utilizadas:} apresentação das ferramentas utilizadas para o desenvolvimento do projeto proposto no trabalho.
     		\item Seção 3, \textbf{Conceitos:} explicação das teorias de Inteligência Artificial e Algoritmo Genético usadas para desenvolver o trabalho.
     		\item Seção 4, \textbf{O Jogo:} descrição em detalhes do jogo, suas variáveis, e de sua implementação.
     		\item Seção 5, \textbf{Resultados:} apresentação dos resultados obtidos no trabalho.
  	 	\end{itemize}
% INTRODUÇÃO - FIM

% DESENVOLVIMENTO - INÍCIO
\FloatBarrier
\newpage % Coloca o conteúdo numa nova página
\section{FERRAMENTAS UTILIZADAS}

	\FloatBarrier
	\subsection{Unity}
		A Unity é um motor de jogo multiplataforma que permite a criação de jogos 2D ou 3D.
		Possui uma interface gráfica que permite desenvolver jogos com facilidade,
		além de ter muitos serviços integrados que aceleram o processo de desenvolvimento.
		As linguagens de programação que podem ser usadas são UnityScript e C\#
		(UNITY, 2016a).
		
		Com mais de 200 jogos na lista de jogos em destaque que foram criados na Unity,
		um dos exemplos que pode-se citar é o \textit{Sky Force},
		um jogo no estilo de ação, missões e "shooter",
		e que está disponível na App Store e Google Play.
		Outro exemplo é o \textit{The Uncertain},
		do qual o jogador é um robô e
		deve resolver quebra-cabeças em uma aventura.
		O jogo está disponível na \textit{Steam}
		(UNITY, 2016b).
		
		O editor da Unity também é extensível, sendo possível implementar funcionalidades ainda não existentes
		(UNITY, 2016a).
		Um exemplo de extensão é o \textit{Rival Theory},
		que implementa vários conceitos de Inteligência Artificial,
		desde funcionalidades básicas como \textit{pathfinding},
		condutas de patrulha, esconder, atacar, seguir, vaguear,
		até conceitos mais complicados como percepção e árvore de condutas
		(RIVAL THEORY, 2015).
		
		\FloatBarrier
		\subsubsection{Interface}
			A interface da Unity é composta por vários painéis chamados \textit{views},
			que podem ser rearranjados, agrupados, separados e fixados
			(UNITY, 2016c).
			O arranjamento padrão das janelas permite o acesso às janelas mais comuns,
			e pode ser visto na Figura \ref{fig:interfaceUnity} a seguir:
			
			\begin{figure}[ht!]
				\caption{A interface padrão da Unity}
				\centering
				\includegraphics[scale=0.5]{InterfaceUnity.jpg}\\
				\vspace{0.5mm}
				\source{Unity.}
				\label{fig:interfaceUnity}
			\end{figure}	
			
			A descrição de cada painel pode ser vista na Tabela \ref{tab:unity} a seguir:
			
			\begin{table}[ht]
				\caption{Descrição dos painéis na Unity}
				\centering
				\small
				\renewcommand{\arraystretch}{1.2} % Aumenta espaçamento vertical
				\begin{tabular}{m{3.1cm} m{11.9cm}}
					\hline 
					Project Window & Exibe a biblioteca de Assets que estão disponíveis para uso no projeto. Ao importar os Assets no projeto, eles aparecerão aqui. \\ 
					\hline 
					Scene View & Permite navegar visualmente e editar a cena. A Scene View pode mostrar em perspectiva 3D ou 2D, dependendo do tipo de projeto que está sendo trabalhando. \\ 
					\hline 
					Hierarchy Window & É uma representação de texto hierárquica de cada objeto na cena. Cada item na cena tem uma entrada na hierarquia, de forma que as duas janelas estão inerentemente conectadas. A hierarquia revela a estrutura da forma como os objectos estão ligados um ao outro. \\ 
					\hline 
					Inspector Window & Permite visualizar e editar todas as propriedades do objeto selecionado. Como diferentes tipos de objetos têm diferentes conjuntos de propriedades, o layout e conteúdo da janela do Inspector pode variar. \\ 
					\hline 
					Toolbar & Fornece acesso aos recursos de trabalho mais essenciais. À esquerda estão as ferramentas básicas para manipular a Scene View e os objetos dentro dela. No centro estão os controles de play, pause e step. Os botões à direita dará acesso aos Serviços da Unity Cloud e da Conta Unity, seguido pelo menu de visibilidade dos layers e, finalmente, o menu de layout do editor (que fornece alguns layouts alternativos para as janelas do editor, e permite salvar um layout customizado). \\ 
					\hline 
				\end{tabular}\\
				\vspace{3mm}
				\source{Unity.}
				\label{tab:unity}
			\end{table}
			
			Os objetos contidos na cena do projeto são chamados de \textit{GameObject},
			sendo que suas características podem ser alteradas por Componentes anexados a ele.
			A Unity possui vários Componentes prontos,
			no entanto é possível criar novos usando as linguagens de programação do qual se dá suporte (C\# e UnityScript).
			Tais Componentes são chamados de Scripts,
			e eles permitem disparar eventos,
			modificar as propriedades de outros Componentes ou do próprio \textit{GameObject} durante o jogo,
			e a interação com o jogador.
			(UNITY, 2016d).

	\FloatBarrier
	\subsection{Visual Studio}
		O Visual Studio é um Ambiente de Desenvolvimento Integrado (IDE) usado para a criação de aplicativos para Windows, Android, iOS, aplicações Web e serviços de nuvem.
		Com ele é possível programar em C\#, Visual Basic, F\#, C++, HTML, JavaScript e Python, dentre outras linguagens de programação. (MICROSOFT, 2016)
		
		A integração do Visual Studio com a Unity, usando C\#, ocorre com a adição do \textit{namespace} UnityEngine ao script,
		%(com o código "using UnityEngine"),
		liberando a utilização da classe \textit{MonoBehaviour},
		da qual possui implementado funcionalidades e funções internas da Unity.
		Para utilizar a classe \textit{MonoBehaviour} nos scripts criados,
		basta estender a classe criada com a \textit{MonoBehaviour} (UNITY, 2016d).
		Desta forma, o cabeçalho do script fica como no código a seguir:
		
		\begin{lstlisting}[language=C++]
using UnityEngine;

public class nomeDaClasse : MonoBehaviour
{
	// Código da classe
}\end{lstlisting}
		
		Com isto feito, é possível utilizar uma das principais vantagens do Visual Studio,
		que é o \textit{AutoComplete}.
		Ou seja, funções internas da Unity, como as do GameObject, aparecem em uma lista conforme se vai digitando,
		facilitando muito a programação do script.
		
		Além da integração com a Unity,
		ainda é possível utilizar as bibliotecas próprias do C\#,
		como bibliotecas matemáticas, acesso à arquivo e listas.
		
		Na \textit{Figura \ref{fig:exvs}} a seguir, é possível ver um exemplo de classe criada no Visual Studio que foi usada neste projeto:
		
		\begin{figure}[ht!]
			\caption{Exemplo de classe no Visual Studio}
			\centering
			\includegraphics[scale=0.4]{InterfaceVisualStudio.png}\\
			\vspace{0.5mm}
			\source{Elaborado pelo autor.}
			\label{fig:exvs}
		\end{figure}
		
		Nota-se que no script de Algoritmo Genético, além do \textit{UnityEngine} e outras duas bibliotecas básicas do C\#,
		também foi utilizada o \textit{System.IO}, uma biblioteca para Leitura e Escrita de Arquivo.
		Qualquer outro script criado para funcionar na Unity tem este padrão.

\FloatBarrier
\newpage % Coloca o conteúdo numa nova página	
\section{CONCEITOS}

	Nesta seção será descrito dois conceitos que foram amplamente utilizados neste trabalho,
	que são Redes Neurais Artificiais (RNA) e Algoritmo Genético (AG).
	Apesar de ser possível descrever várias variações de aplicação desses conceitos,
	será explicado apenas a essência deles,
	e o que foi necessário para desenvolver este trabalho.

	\FloatBarrier
	\subsection{Redes Neurais Artificiais}
	% 1 Definição
	As Redes Neurais Artificiais são uma família de modelos inspirados no funcionamento do cérebro humano,
	sendo que a "baixo nível"{} procuram imitar o que acontece nos neurônios.
	São usadas para estimar ou aproximar funções que dependam de um grande número de entradas.
	
	\subsubsection{Funcionamento}
	% 2 Arquitetura
	% 2.1 Funções de entrada e ativação
	A unidade em uma RNA é o neurônio.
	O neurônio recebe estímulos, que na RNA são números reais.
	Recebendo os vários estímulos, o neurônio faz uma operação com todos os números recebidos,
	que normalmente é um somatório.
	Esta é a função de entrada.
	
	Após aplicar a função de entrada em todos os estímulos,
	é obtido um único valor.
	Este valor, então, é processado por uma função de ativação.
	Existem várias funções de ativação que podem ser utilizadas,
	e pode-se ver alguns exemplos na Figura \ref{fig:activationFunctions}.
	
	\begin{figure}[ht!]
		\caption{Funções de Ativação comuns em uma RNA}
		\centering
		\includegraphics[scale=0.3]{ActivationFunctions.png}\\
		\vspace{0.5mm}
		\source{Turing Finance.}
		\label{fig:activationFunctions}
	\end{figure}
	
	Diferentes funções de ativação se adequam melhor dependendo da aplicação da RNA,
	assim como a camada em que o neurônio se encontra.
	
	O resultado da função de ativação é então multiplicado por um peso,
	e passado para o próximo neurônio como um estímulo.
	
	% 2.2 Particularidades (camadas existentes, papel de cada uma)
	A arquitetura de uma RNA é formada por uma camada de entrada (input layer),
	uma camada de saída (output layer),
	e as camadas ocultas (hidden layers)
	que poderão conter de zero a muitas camadas.
	Cada camada na RNA terá \textit{n} neurônios,
	sendo \textit{n} qualquer valor maior ou igual a 1.
	Um exemplo de Rede Neural Artificial com uma camada oculta pode ser visto na Figura \ref{fig:basicnn}.
	
	\begin{figure}[ht!]
		\caption{RNA com uma camada oculta}
		\centering
		\includegraphics[scale=0.5]{BasicNN.jpg}\\
		\vspace{0.5mm}
		\source{Wikimedia.}
		\label{fig:basicnn}
	\end{figure}
	
	Como ilustrado na Figura \ref{fig:basicnn},
	cada neurônio de uma camada se conecta com todos os outros neurônios da camada seguinte.
	Existem muitas variações de arquiteturas de RNA,
	não sendo todas que seguem o padrão visto na Figura \ref{fig:basicnn},
	porém esta é a arquitetura mais "tradicional"{}.
	
	Uma Rede Neural Artificial sempre terá uma camada de entrada e uma de saída.
	No entanto, o número de camadas ocultas pode variar bastante entre um problema e outro.
	Estas camadas ocultas são explicadas como sendo "extratoras de características"{}
	(STACKEXCHANGE, 2013).
	
	Então, se o número de camadas ocultas varia,
	de que maneira pode-se determinar o número de camadas ocultas a serem utilizadas em um determinado problema?
	Caso os dados sejam linearmente separáveis,
	não é necessário nenhuma camada oculta.
	Caso contrário, se forem dados não lineares,
	uma ou mais camadas ocultas serão necessárias
	(STACKEXCHANGE, 2013).
	Quanto mais camadas ocultas forem utilizadas,
	maior a granularidade dos resultados obtidos na RNA,
	como pode ser visto na Figura \ref{fig:nnhiddenlayer}.
	
	\begin{figure}[ht!]
		\caption{Importância da camada oculta}
		\centering
		\includegraphics[scale=1.5]{HiddenLayers.png}\\
		\vspace{0.5mm}
		\source{Stackoverflow.}
		\label{fig:nnhiddenlayer}
	\end{figure}
	
	Deve-se lembrar que, apesar de muitas camadas ocultas aumentar a granularidade dos resultados,
	isto vem com um custo:
	mais tempo de processamento e
	influência no tempo de aprendizagem.
	É por este motivo que deve-se buscar a arquitetura otimizada ao problema,
	com baixo tempo de processamento e alto desempenho.
	Outro detalhe importante é a necessidade de colocar funções de ativação não-lineares nas camadas ocultas,
	uma vez que são elas que garantem a não-linearidade dos resultados
	(STACKEXCHANGE, 2013).
	
	Existem muitas discussões a respeito do número de camadas ocultas a serem usadas, caso ela seja necessária,
	mas um consenso que existe é que uma única camada é o suficiente para a maioria dos problemas
	(STACKEXCHANGE, 2013).
	
	Considerando a necessidade de colocar uma camada oculta na RNA,
	existem regras empíricas que ajudam a escolher um número de neurônios a serem colocados na camada oculta.
	Uma delas é encontrar a média do número de neurônios de entrada e de saída,
	e usar este valor como ponto de partida,
	ajustando-o então com testes
	(STACKEXCHANGE, 2013).
	
	\subsubsection{Tipos de aprendizado}
	% 3 Tipos de aprendizado (com exemplos de utilização)
	Um dos maiores atrativos de uma RNA é sua capacidade de aprender,
	e atualmente existem três paradigmas de aprendizagem:
	Aprendizado Supervisionado, Aprendizado Não-supervisionado e Aprendizado Reforçado.
	
	No Aprendizado Supervisionado,
	tem-se os dados de entrada, e infere-se os dados de saída.
	Sabendo-se estas duas informações, treina-se a RNA para que produza os resultados esperados,
	ajustando os pesos de suas conexões.
	Esta forma de aprendizagem é utilizada principalmente para duas finalidades:
	reconhecimento de padrões e regressão de funções.
	
	Dependendo da finalidade da RNA no Aprendizado Supervisionado,
	diferentes funções de ativação são recomendadas na camada de saída
	(as funções de ativação podem ser vistas na Figura \ref{fig:activationFunctions}, página \pageref{fig:activationFunctions}).
	Caso a tarefa da RNA seja de reconhecer padrões,
	as funções recomendadas são a step (b), sigmoide (d) e tangente hiperbólica (e).
	Por outro lado, se a tarefa for de regressão de uma função,
	a função linear (a) é a mais recomendada (RESEARCHGATE, 2013).
	
	No Aprendizado Não-supervisionado,
	tem-se os dados de entrada e uma função de custo a ser minimizada.
	Com o decorrer das iterações,
	os pesos da RNA serão adaptados para minimizar o custo.
	Tarefas que usam tal paradigma são os que envolvem problemas de estimativa.
	
	No Aprendizado Reforçado,
	os dados de entrada são desconhecidos,
	e o cálculo do custo é feita de maneira dinâmica.
	Os dados de entrada são gerados através das interações do RNA com o ambiente.
	Exemplos de utilização deste paradigma são
	programação dinâmica e problemas de controle.

	\FloatBarrier
	\subsection{Algoritmo Genético}
	% 1 Definição
	Algoritmo genético faz parte da computação evolutiva,
	e inspirou-se na teoria de Darwin sobre a evolução das espécies.
	É uma busca heurística usada para encontrar soluções de otimização e resolver problemas de busca.
	
	\FloatBarrier
	\subsubsection{Conceitos de AG}
	% 2 Conceitos
	% 2.1 Cromossomos
	A unidade que caracteriza um determinado indivíduo na população é o cromossomo.
	% 2.1.1 Gene
	O cromossomo é constituído de vários genes,
	que carregam uma informação ou característica de um indivíduo.
	% 2.1.2 Representação do cromossomo
	O gene pode ser representado de várias formas dentro do AG,
	sendo a mais simples delas a representação binária,
	como pode ser visto na Tabela \ref{tab:cromosome}.
	
	\begin{table}[ht]
		\caption{Representação binária de cromossomos}
		\centering
		\begin{tabular}{c c c c c c c c c c c}
			\hline 
			0 & 1 & 1 & 0 & 1 & 0 & 0 & 0 & 1 & 0 & 1\\ 
			\hline 
			1 & 0 & 1 & 0 & 0 & 1 & 1 & 0 & 0 & 1 & 0\\ 
			\hline 
		\end{tabular} \\
		\vspace{3mm}
		\source{Elaborado pelo autor.}
		\label{tab:cromosome}
	\end{table}
	
	Outras representações de cromossomos também existem.
	Por exemplo, existe a codificação por permutação,
	utilizado em problemas de ordenação.
	Neste caso, todo cromossomo é um vetor de números,
	e cada número representa uma posição.
	
	Outro exemplo é codificação por valor.
	Este valor poderá ser qualquer coisa relacionada ao problema,
	como números reais ou sequência de caracteres.
	E como último exemplo existe a codificação em árvore,
	em que cada cromossomo é uma árvore de objetos,
	como funções ou comandos em uma linguagem de programação.
	
	% 2.2 Indivíduo
	Um indivíduo no AG sempre possuirá um cromossomo.
	Este indivíduo carrega uma das possíveis soluções do problema.
	%e pode ser uma solução boa ou não.
	
	% 2.3 População
	O indivíduo faz parte de uma população.
	%Logo, pode-se dizer que uma população é um conjunto de soluções.
	% 2.3.1 Tamanho de população
	Para se resolver um problema usando AG,
	nem sempre uma população enorme irá se traduzir em uma convergência mais rápida.
	De acordo com algumas pesquisas,
	população em torno de 30 indivíduos se mostram ser eficientes,
	apesar de que outros valores possam ser melhores dependendo do problema
	(OBITKO, 1998).
	
	% 2.4 Geração
	Uma geração diz respeito à iteração do qual a população se encontra.
	Uma população nova criada a partir de uma população antiga é uma nova geração.
	Tem-se, então, várias gerações de populações diferentes (ou evoluídas).
	
	% 2.5 Avaliação
	% 2.5.1 Fitness (aptidão)
	Durante o algoritmo do AG, deve-se calcular o o fitness (nível de aptidão) de cada indivíduo da população.
	A forma de se calcular o fitness do indivíduo depende do problema,
	e este cálculo faz parte do processo de Avaliação da população,
	que pode ser visto em mais detalhes na Tabela \ref{tab:agCycle} e Figura \ref{fig:gaflow} à frente.
	
	% 2.6 Seleção
	No processo de Seleção,
	indivíduos da população são selecionados para passarem para a próxima geração ou se reproduzirem.
	A Seleção usa o valor de fitness para dar maiores chances de selecionar os indivíduos mais adaptados.
	
	% 2.6.1 Roleta
	Um método comum para se selecionar indivíduos é o método da Roleta.
	Neste método, um intervalo numérico relativo ao valor de fitness é atribuído a cada indivíduo.
	Então é sorteado aleatoriamente um valor no intervalo de todos os indivíduos.
	O indivíduo que conter o número sorteado dentro de seu intervalo é o escolhido para passar para a próxima geração ou se reproduzir.
	Percebe-se assim que indivíduos com maior fitness terão um intervalo maior e, portanto, maiores chances de serem selecionados.
	
	% 2.6.2 Ranking
	Outro exemplo de método de Seleção é o de Ranking.
	De forma similar ao da Roleta, é usado o valor de fitness.
	No entanto, o diferencial está no fato de ordenar os indivíduos em um ranking,
	e atribuir um intervalo numérico dependendo do ranking do indivíduo
	(quanto maior o ranking, maior o intervalo atribuído).
	
	% 2.7 Crossover
	Após a seleção existe o Crossover (ou cruzamento).
	%análogo à reprodução que se estuda em biologia.
	Se for decidido que haverá um crossover,
	informações do cromossomo de 2 indivíduos são misturados,
	criando "cromossomos filhos"{}.
	Neste processo, são selecionados pontos nos cromossomos pais,
	e os cromossomos filhos copiam do primeiro cromossomo pai até aquele ponto,
	passando então a copiar do segundo cromossomo pai.
	O crossover pode ter um único ponto ou vários pontos de crossover.
	Também existe o crossover uniforme,
	em que é copiado vários pequenos trechos de ambos os pais,
	com taxa de 50\% de se copiar de cada um deles.
	
	% 2.7.1 Taxa de Crossover
	No entanto, o crossover não necessariamente ocorre sempre.
	Existe uma probabilidade dele acontecer,
	e essa probabilidade é a taxa de crossover.
	Considerando-se, por exemplo, uma taxa de crossover de 90\%,
	tem-se que 90\% dos indivíduos selecionados passarão pelo processo de crossover,
	e que 10\% serão copiados para a nova população.
	
	% 2.8 Mutação
	Após o crossover,
	os cromossomos passam pelo processo de Mutação.
	A mutação permite a mudança aleatória de alguns genes escolhidos aleatoriamente.
	% 2.8.1 Taxa de Mutação
	A chance de ocorrer uma mutação em um determinado gene é determinado pela taxa de mutação.
	Por exemplo, se a taxa de mutação for de 1\%,
	isto significa que 1\% dos genes terão seus valores alterados.
	
	% 2.9 Elitismo
	A fim de não se perder bons resultados por causa do Crossover e Mutação,
	uma técnica muito aplicada e eficiente é a do Elitismo.
	Nesta técnica, o melhor (ou melhores) indivíduos da população são copiados sem nenhuma alteração para a próxima geração.
	
	\FloatBarrier
	\subsubsection{Processo}
	% 3 Processo
	O Algoritmo Genético mais simples de ser representado tem 5 etapas,
	em um processo que se repete até que uma determinada condição de parada seja satisfeita.
	A melhor condição para se parar é uma boa "avaliação"{} (fitness) dos indivíduos,
	mas também é possível parar por um limite de gerações.
	
	\begin{table}[h]
		\caption{Ciclo do Algoritmo Genético}
		\centering
		\small
		\renewcommand{\arraystretch}{1.2} % Aumenta espaçamento vertical
		\begin{tabular}{>{\centering\arraybackslash}m{3.5cm} m{11.5cm}}
			\hline 
			\textbf{Inicia População} & Gera uma população aleatória de n cromossomos. \\ 
			\hline 
			\textbf{Avaliação} & Testa o valor de fitness f(x) de cada cromossomo x da população. Se a condição de parada for satisfeita, o algoritmo acaba. \\ 
			\hline 
			\textbf{Seleção} & Seleciona 2 cromossomos de acordo com o fitness deles, ou seja, quanto maior o valor de fitness, maior a probabilidade. \\ 
			\hline 
			\textbf{Crossover} & Copia partes dos cromossomos dos pais em cromossomos filhos. \\ 
			\hline 
			\textbf{Mutação} & Aplica a probabilidade de alterar os genes dos indivíduos da nova população. \\ 
			\hline 
			\textbf{Loop} & Retorna ao passo de Avaliação. \\ 
			\hline 
		\end{tabular}\\
		\vspace{3mm}
		\source{Elaborado pelo autor.}
		\label{tab:agCycle}
	\end{table}
	
	\begin{figure}[ht!]
		\centering
		\caption{Passo a passo do Algoritmo Genético}
		\includegraphics[scale=0.8]{GeneticAlgorithmFlow.png}\\
		\vspace{0.5mm}
		\source{Tech-Effigy.}
		\label{fig:gaflow}
	\end{figure}
	
	A Figura \ref{fig:gaflow} ilustra o que foi descrito.
	O ciclo de um Algoritmo Genético pode ser representado como na Tabela \ref{tab:agCycle}, na ordem apresentada.
	A maior motivação de se utilizar um AG é que, a cada geração,
	as soluções ficam melhores.
	
	\FloatBarrier
	\subsubsection{Exemplos de aplicação}
	% 4 Exemplos de aplicação
	<Exemplos de aplicação do Algoritmo Genético>
	
	\FloatBarrier
	\subsubsection{Variação com RNA}
	No trabalho aqui desenvolvido,
	Redes Neurais Artificiais e Algoritmos Genéticos são utilizados em conjunto.
	Na Figura \ref{fig:nn-ag} a seguir é possível visualizar como é implementação de ambos os conceitos juntos.
	
	\begin{figure}[ht!]
		\centering
		\caption{RNA e AG juntos}
		\includegraphics[scale=0.7]{NeuralNetworkIntoChromosome.png}\\
		\vspace{0.5mm}
		\source{Evo-neural-network-agents.}
		\label{fig:nn-ag}
	\end{figure}
	
	Os pesos da RNA são colocados na estrutura do cromossomo,
	e então é tratado como um cromossomo para realizar o processo do AG.
	Ao finalizar, o cromossomo retorna à RNA como pesos,
	usa-se e repete o processo.

\FloatBarrier
\newpage % Coloca o conteúdo numa nova página
\section{O JOGO}
	
	\FloatBarrier
	\subsection{Descrição do jogo}
	<Descrever o jogo, suas variáveis, complexidades, particularidades, etc>
	
	\FloatBarrier
	\subsection{Rede Neural Artificial Implementado}
	<Descrever em detalhes qual foi a Rede Neural Artificial implementada>
	
	\FloatBarrier
	\subsection{Algoritmo Genético Implementado}
	<Descrever em detalhes qual foi o Algoritmo Genético implementado>
	
	\FloatBarrier
	\subsection{Funcionamento do jogo}
	<Descrever como ocorre o funcionamento do jogo como um todo, dado que o leitor sabe os detalhes de cada parte.>

% DESENVOLVIMENTO - FIM

% CONCLUSÃO - INÍCIO
\FloatBarrier
\newpage % Coloca o conteúdo numa nova página
\section{RESULTADOS}
	<Inserir gráficos de evolução do algoritmo, imagens do jogo, explicar desempenho da aprendizagem, etc.>
% CONCLUSÃO - FIM

\FloatBarrier
\newpage % Coloca o conteúdo numa nova página
\section*{\hfil REFERÊNCIAS}
\addcontentsline{toc}{section}{REFERÊNCIAS} % Adiciona as referências no Sumário
	\singlespace
	UNITY. Game engine, tools and multiplatform. 2016a. Disponível em: \textless \url{https://unity3d.com/pt/unity}\textgreater. Acesso em 01 Julho 2016.\par
	UNITY. Made with Unity - Games. 2016b. Disponível em: \textless \url{https://madewith.unity.com/games}\textgreater. Acesso em 01 Julho 2016.\par
	UNITY. Made with Unity - Manual: Learning the Interface. 2016c. Disponível em: \textless \url{https://docs.unity3d.com/Manual/LearningtheInterface.html}\textgreater. Acesso em 05 Julho 2016.\par
	UNITY. Made with Unity - Manual: Creating and Using Scripts. 2016d. Disponível em: \textless \url{https://docs.unity3d.com/Manual/CreatingAndUsingScripts.html}\textgreater. Acesso em 05 Julho 2016.\par
	RIVAL THEORY. Features. 2015. Disponível em: \textless \url{http://rivaltheory.com/rain/features/}\textgreater. Acesso em 01 Julho 2016.\par
	MICROSOFT. Free Dev Tools - Visual Studio Community 2015. 2016. Disponível em: \textless \url{https://www.visualstudio.com/en-us/products/visual-studio-community-vs.aspx}\textgreater. Acesso em 07 Julho 2016.\par
	CNBC. Digital gaming sales hit record \$61 billion in 2015: Report. Disponível em: \textless \url{http://www.cnbc.com/2016/01/26/digital-gaming-sales-hit-record-61-billion-in-2015-report.html}\textgreater. Acesso em 11 Julho 2016.\par
	YANNAKAKIS, Georgios N; TOGELIUS, Julian. A Panorama of Artificial and Computational Intelligence in Games. Disponível em: \textless \url{http://julian.togelius.com/Yannakakis2014Panorama.pdf}\textgreater. Acesso em 11 Julho 2016.\par
	AIGAMEDEV. Top 10 Most Influential AI Games. Disponível em: \textless \url{http://aigamedev.com/open/highlights/top-ai-games/}\textgreater. Acesso em 12 Julho 2016.\par
	ORKIN, Jeff. Three States and a Plan: The A.I. of F.E.A.R. Disponível em: \textless \url{http://alumni.media.mit.edu/~jorkin/gdc2006_orkin_jeff_fear.pdf}\textgreater. Acesso em 12 Julho 2016.\par
	STACKEXCHANGE. How to choose the number of hidden layers and nodes in a feedforward neural network?. Disponível em: \textless \url{https://stats.stackexchange.com/questions/181/how-to-choose-the-number-of-hidden-layers-and-nodes-in-a-feedforward-neural-netw}\textgreater. Acesso em 13 Julho 2016.\par
	STACKEXCHANGE. What does the hidden layer in a neural network compute?. Disponível em: \textless \url{https://stats.stackexchange.com/questions/63152/what-does-the-hidden-layer-in-a-neural-network-compute}\textgreater. Acesso em 13 Julho 2016.\par
	STACKOVERFLOW. Role of Bias in Neural Networks. Disponível em: \textless \url{https://stackoverflow.com/questions/2480650/role-of-bias-in-neural-networks}\textgreater. Acesso em 13 Julho 2016.\par
	OBITKO, Marek. Introduction to Genetic Algorithms. Disponível em: \textless \url{http://www.obitko.com/tutorials/genetic-algorithms/index.php}\textgreater. Acesso em 13 Julho 2016.\par
	RESEARCHGATE. How to select the best transfer function for a neural network model?. Disponível em: \textless \url{https://www.researchgate.net/post/How_to_select_the_best_transfer_function_for_a_neural_network_model}\textgreater. Acesso em 13 Julho 2016.\par
	Wikimedia. File:Colored neural network.svg. Disponível em: \textless \url{https://commons.wikimedia.org/wiki/File:Colored_neural_network.svg}\textgreater. Acesso em 13 Julho 2016.\par
	STACKOVERFLOW. How do you decide the parameters of a Convolutional Neural Network for image classification?. Disponível em: \textless \url{https://stackoverflow.com/questions/24509921/how-do-you-decide-the-parameters-of-a-convolutional-neural-network-for-image-cla}\textgreater. Acesso em 13 Julho 2016.\par
	Tech-Effigy. The Genetic Algorithm - Explained. Disponível em: \textless \url{https://techeffigytutorials.blogspot.com.br/2015/02/the-genetic-algorithm-explained.html}\textgreater. Acesso em 14 Julho 2016.\par
	Evo-neural-network-agents. Simulation of evolution of neural network driven agents in the small world with the pieces of food. Disponível em: \textless \url{https://github.com/lagodiuk/evo-neural-network-agents}\textgreater. Acesso em 14 Julho 2016.\par
	Turing Finance. 10 misconceptions about Neural Networks. Disponível em: \textless \url{http://www.turingfinance.com/misconceptions-about-neural-networks/}\textgreater. Acesso em 14 Julho 2016.\par

\end{document}
