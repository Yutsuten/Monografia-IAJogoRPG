% Tamanho do texto, tipo de papel e documento
\documentclass[12pt,a4paper]{article}

% PACOTES
% Padrão
\usepackage[utf8]{inputenc}
\usepackage{amsmath}
\usepackage{amsfonts}
\usepackage{amssymb}

% Pacotes para alterar fonte
\usepackage[T1]{fontenc}
\usepackage{uarial}
% Personaliza lista de itens
\usepackage{enumitem}
% Permite adicionar imagens
\usepackage{graphicx}
% Pacote para referências
\usepackage{csquotes}
% Coloca alguns termos do artigo em português (figura, tabela)
\usepackage[brazilian]{babel}
% Permite alterar distância entre linhas
\usepackage{setspace}
% Colocando indentação em todos os parágrafos (por default o primeiro parágrafo da seção não tem)
\usepackage{indentfirst}
% Margens nas páginas
\usepackage[top=3cm, left=3cm, bottom=2cm, right=2cm]{geometry}
% Links do sumário para as respectivas páginas
\usepackage[colorlinks=true,linkcolor=black]{hyperref}

% CONFIGURAÇÃO DE VARIÁVEIS
% Mostra numero da pagina no canto superior direito
\pagestyle{myheadings}
% Indentação dos parágrafos
\parindent 30pt
% Distância entre linhas
\onehalfspacing
% Acessando comandos internos
\makeatletter
	% Colocando espaço de "um espaço" entre o número e o título da seção (default é espaço de 1 quad (1em))
	\renewcommand{\@seccntformat}[1]{\csname the#1\endcsname\ }
	% Deixa os itens do sumário com pontos
	\renewcommand*\l@section{\@dottedtocline{1}{1.5em}{2.3em}}
\makeatother
% Editando títulos dos índices
\addto\captionsbrazilian{
	\renewcommand{\listfigurename}{\centering LISTA DE FIGURAS}
	\renewcommand{\listtablename}{\centering LISTA DE TABELAS}
	\renewcommand{\contentsname}{\centering SUMÁRIO}
}
% Escolhendo a fonte 
\renewcommand{\familydefault}{\sfdefault}
% Estilo de listas
\renewcommand{\theenumi}{\alph{enumi}}


% INFORMAÇÕES DO TRABALHO
\author{MATEUS GONÇALEZ ETTO}
\title{UTILIZAÇÃO DE INTELIGÊNCIA ARTIFICIAL EM JOGO RPG}
\date{\today}


% COMANDOS PERSONALIZADOS
% Fonte da imagem
\newcommand{\source}[1]{Fonte: {#1}}
% Citações longas
\def\longcitation{\list{}{\vspace{-6mm} \leftmargin4.0cm \singlespace \small}\item[]}
\let\endlongcitation=\endlist
% Capa: objetivo do trabalho
\def\articleobjective{\list{}{\vspace{1.5cm} \leftmargin7.0cm \singlespace \small}\item[]}
\let\endarticleobjective=\endlist


% INICIO DO ARTIGO
\begin{document}

% CAPA - INÍCIO
\thispagestyle{empty} % Oculta o número da página
\begin{center}
\makeatletter
	\textbf{UNIVERSIDADE ESTADUAL PAULISTA}\\
	\textbf{FACULDADE DE CIÊNCIAS}\\
	\textbf{DEPARTAMENTO DE COMPUTAÇÃO}\\
	\textbf{BACHARELADO EM CIÊNCIA DA COMPUTAÇÃO}\\

	\vspace{4.5cm} % Adicionando espaço extra
	\textbf{\@author}

	\vspace{4.5cm} % Adicionando espaço extra
	\textbf{\large \@title}
	
	\vspace*{\fill} % Adicionando espaço para escrever em baixo
	BAURU – SP\\
	\the\year
\makeatother
\end{center}
% CAPA - FIM

% FOLHA DE ROSTO - INÍCIO
\clearpage % Coloca o conteúdo numa nova página
\thispagestyle{empty} % Oculta o número da página
\begin{center}
\makeatletter
	\textbf{\@author}

	\vspace{4.5cm} % Adicionando espaço extra
	\textbf{\large \@title}
\makeatother
\end{center}

\begin{articleobjective}
	Trabalho de Conclusão de Curso de graduação apresentado à disciplina Projeto e Implementação de Sistemas do curso de Bacharelado em Ciência da Computação da Faculdade de Ciências da Universidade Estadual Paulista "Júlio de Mesquita Filho" como requisito para obtenção do título de Bacharel em Ciência da Computação.
	
	\vspace{1.0cm} % Adicionando espaço extra
	Orientadora: Profa. Dra. Simone das Graças Domingues Prado
\end{articleobjective}

\begin{center}
	\vspace*{\fill} % Adicionando espaço para escrever em baixo
	BAURU – SP\\
	\the\year
\end{center}
% FOLHA DE ROSTO - FIM

% FOLHA DE APROVAÇÃO - INÍCIO
\clearpage % Coloca o conteúdo numa nova página
\thispagestyle{empty} % Oculta o número da página
\begin{center}
\makeatletter
	\textbf{\@author}

	\vspace{3.0cm} % Adicionando espaço extra
	\textbf{\large \@title}
\makeatother
\end{center}

\begin{articleobjective}
	Trabalho de Conclusão de Curso de graduação apresentado à disciplina Projeto e Implementação de Sistemas do curso de Bacharelado em Ciência da Computação da Faculdade de Ciências da Universidade Estadual Paulista "Júlio de Mesquita Filho" como requisito para obtenção do título de Bacharel em Ciência da Computação.
\end{articleobjective}

\begin{center}
	\vspace{1.0cm}
	BANCA EXAMINADORA\\
	
	\vspace{1.0cm}
	\underline{\hspace{8cm}}\\
	Profa. Dra. Simone das Graças Domingues Prado\\
	\vspace{1.0cm}
	\underline{\hspace{8cm}}\\
	(Nome do segundo membro da Banca Examinadora)\\
	\vspace{1.0cm}
	\underline{\hspace{8cm}}\\
	(Nome do terceiro membro da Banca Examinadora)

	\vspace*{\fill} % Adicionando espaço para escrever em baixo
	Bauru, \underline{\hspace{1cm}} de \underline{\hspace{3cm}} de \underline{\hspace{1.5cm}}
\end{center}
% FOLHA DE APROVAÇÃO - FIM

% RESUMO - INÍCIO
\clearpage % Coloca o conteúdo numa nova página
\thispagestyle{empty} % Oculta o número da página
\section*{\hfil RESUMO} % \hfil centraliza o título
	\singlespace
	Aqui se encontrará o resumo de minha Monografia.
	A fonte que está sendo usada é Arial, com tamanho 12.
	Em citações longas, notas de rodapé, referências, resumo (aqui!) e abstract usarei espaçamento simples entrelinhas (espaço 1) conforme descrito nos slides sobre ABNT passados em sala de aula (ppt ABNT-NBR 14724-2005.pdf).
	O limite de palavras no resumo é de 500 palavras.
	
	\vspace{0.8cm}
	\textbf{PALAVRAS-CHAVE:} palavra1, palavra2, palavra3
% RESUMO - FIM

\onehalfspacing

% ABSTRACT - INÍCIO
\clearpage % Coloca o conteúdo numa nova página
\thispagestyle{empty} % Oculta o número da página
\section*{\hfil ABSTRACT} % \hfil centraliza o título
	\singlespace
	Here you will find the abstract of my monograph.
	The font being used is Arial size 12.
	In long citations, footnotes, references, summary and abstract (here!) will use single line spacing (space 1) as described in the slides about ABNT given in the classroom (ppt ABNT-NBR 14724-2005.pdf).
	The limit of words in the abstract is 500 words.
	
	\vspace{0.8cm}
	\textbf{KEY-WORDS:} word1, word2, word3
% ABSTRACT - FIM

\onehalfspacing

% LISTA DE FIGURAS - INÍCIO
\clearpage % Coloca o conteúdo numa nova página
\thispagestyle{empty} % Oculta o número da página
\listoffigures % Cria a lista de figuras
% LISTA DE FIGURAS - FIM

% LISTA DE TABELAS - INÍCIO
\clearpage % Coloca o conteúdo numa nova página
\thispagestyle{empty} % Oculta o número da página
\listoftables % Cria a lista de tabelas
% LISTA DE TABELAS - FIM

% SUMÁRIO - INÍCIO
\clearpage % Coloca o conteúdo numa nova página
\thispagestyle{empty} % Oculta o número da página
\tableofcontents % Cria o Sumário
% SUMÁRIO - FIM

% INTRODUÇÃO - INÍCIO
\clearpage % Coloca o conteúdo numa nova página
\section{INTRODUÇÃO}
	<Escrever sobre a influência dos jogos na sociedade e mercado.>
	
	<Escrever sobre a influência de IA nos jogos.>

	\subsection{Objetivos do Trabalho}
	
		\subsubsection{Objetivo Geral}
			Produzir um jogo RPG em turnos que implementa conceitos avançados de Inteligência Artificial,
			sendo que esta inteligência é capaz de tomar decisões de forma autônoma sobre o que deve fazer.
		
		\subsubsection{Objetivos Específicos}
			\begin{enumerate}[noitemsep]
				\item Criar um jogo razoavelmente complexo.
				\item Criar uma Inteligência Artificial capaz de jogar o jogo tão bem quanto um ser humano.
				\item Criar uma IA capaz de aprender conforme joga.
			\end{enumerate}			
	
	\subsection{Organização da Monografia}
		Este trabalho está dividido em 5 seções, sendo esta seção (Introdução) a primeira. As outras seções são:
		\begin{itemize}[noitemsep]
     		\item Seção 2, \textbf{Ferramentas Utilizadas:} apresentação das ferramentas utilizadas para o desenvolvimento do projeto proposto no trabalho.
     		\item Seção 3, \textbf{Conceitos:} explicação das teorias de Inteligência Artificial e Algoritmo Genético usadas para desenvolver o trabalho.
     		\item Seção 4, \textbf{O Jogo:} descrição em detalhes do jogo, suas variáveis, e de sua implementação.
     		\item Seção 5, \textbf{Resultados:} apresentação dos resultados obtidos no trabalho.
  	 	\end{itemize}
% INTRODUÇÃO - FIM

% DESENVOLVIMENTO - INÍCIO
\clearpage % Coloca o conteúdo numa nova página
\section{FERRAMENTAS UTILIZADAS}

	\subsection{Unity}
		<Escrever o que é a Unity, suas funcionalidades e facilidades>

	\subsection{Visual Studio}
		<Escrever o que é o Visual Studio, o papel dele na implementação do projeto, integração com a Unity (autocomplete), etc>
	
\section{CONCEITOS}

	\subsection{Redes Neurais}
	<Escrever teorias de Redes Neurais, é provavelmente é possível copiar da Fundamentação Teórica para já ter uma base.>

	\subsection{Algoritmo Genético}
	<Escrever teorias de Algoritmo Genético, é provavelmente é possível copiar da Fundamentação Teórica para já ter uma base.>

\section{O JOGO}
	
	\subsection{Descrição do jogo}
	<Descrever o jogo, suas variáveis, complexidades, particularidades, etc>
	
	\subsection{Rede Neural Artificial Implementado}
	<Descrever em detalhes qual foi a Rede Neural Artificial implementada>
	
	\subsection{Algoritmo Genético Implementado}
	<Descrever em detalhes qual foi o Algoritmo Genético implementado>
	
	\subsection{Funcionamento do jogo}
	<Descrever como ocorre o funcionamento do jogo como um todo, dado que o leitor sabe os detalhes de cada parte.>

% DESENVOLVIMENTO - FIM

% CONCLUSÃO - INÍCIO
\clearpage % Coloca o conteúdo numa nova página
\section{RESULTADOS}
	<Inserir gráficos de evolução do algoritmo, imagens do jogo, explicar desempenho da aprendizagem, etc.>

\clearpage % Coloca o conteúdo numa nova página
\section*{REFERÊNCIAS}
\addcontentsline{toc}{section}{REFERÊNCIAS} % Adiciona as referências no Sumário
	\singlespace
	<Inserir aqui as referências.>
% CONCLUSÃO - FIM

\end{document}
